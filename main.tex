%%%%%%%%%%%%%%%%%
% This is an example CV created using altacv.cls (v1.1.2, 1 February 2017) written by
% LianTze Lim (liantze@gmail.com), based on the 
% Cv created by BusinessInsider at http://www.businessinsider.my/a-sample-resume-for-marissa-mayer-2016-7/?r=US&IR=T
% 
%% It may be distributed and/or modified under the
%% conditions of the LaTeX Project Public License, either version 1.3
%% of this license or (at your option) any later version.
%% The latest version of this license is in
%%    http://www.latex-project.org/lppl.txt
%% and version 1.3 or later is part of all distributions of LaTeX
%% version 2003/12/01 or later.
%%%%%%%%%%%%%%%%

%% If you want to use \orcid or the
%% academicons icons, add "academicons"
%% to the \documentclass options. 
%% Then compile with XeLaTeX or LuaLaTeX.
% \documentclass[10pt,a4paper,academicons]{altacv}
\documentclass[10pt,a4paper]{altacv}

%% AltaCV uses the fontawesome and academicon fonts
%% and packages. 
%% See texdoc.net/pkg/fontawecome and http://texdoc.net/pkg/academicons for full list of symbols.
%% When using the "academicons" option,
%% Compile with LuaLaTeX for best results. If you
%% want to use XeLaTeX, you may need to install
%% Academicons.ttf in your operating system's font %% folder.


% Change the page layout if you need to
\geometry{left=1cm,right=9cm,marginparwidth=6.8cm,marginparsep=1.2cm,top=1cm,bottom=1cm}

% Change the font if you want to.

% If using pdflatex:
\usepackage[utf8]{inputenc}
\usepackage[T1]{fontenc}
\usepackage[default]{lato}

% If using xelatex or lualatex:
% \setmainfont{Lato}

% Change the colours if you want to
\definecolor{VividPurple}{HTML}{3E0097}
\definecolor{SlateGrey}{HTML}{2E2E2E}
\definecolor{LightGrey}{HTML}{666666}
\colorlet{heading}{VividPurple}
\colorlet{accent}{VividPurple}
\colorlet{emphasis}{SlateGrey}
\colorlet{body}{LightGrey}

% Change the bullets for itemize and rating marker
% for \cvskill if you want to
\renewcommand{\itemmarker}{{\small\textbullet}}
\renewcommand{\ratingmarker}{\faCircle}

%% sample.bib contains your publications
\addbibresource{sample.bib}

\begin{document}
\name{Rohit P. Tahiliani}
\tagline{Postgraduate Student at Trinity College Dublin}
% Cropped to square from https://en.wikipedia.org/wiki/Marissa_Mayer#/media/File:Marissa_Mayer_May_2014_(cropped).jpg, CC-BY 2.0
%\photo{2.5cm}{rohitt}
\personalinfo{%
  % Not all of these are required!
  % You can add your own with \printinfo{symbol}{detail}
  \email{tahiliar@tcd.ie}
  \phone{+353-899882684}
  %\mailaddress{Address, Street, 00000 County}
  \location{Dublin, IE}
  %\homepage{marissamayr.tumblr.com/}
  %\twitter{@marissamayer}
  \linkedin{linkedin.com/in/rohittahiliani}
   \github{github.com/sneaker-rohit} % I'm just making this up though.
%   \orcid{orcid.org/0000-0000-0000-0000} % Obviously making this up too. If you want to use this field (and also other academicons symbols), add "academicons" option to \documentclass{altacv}
}

%% Make the header extend all the way to the right, if you want. Extend the right margin by 8cm (=6.8cm marginparwidth + 1.2cm marginparsep)
\begin{adjustwidth}{}{-8cm}
\makecvheader
\end{adjustwidth}

%% Provide the file name containing the sidebar contents as an optional parameter to \cvsection.
%% You can always just use \marginpar{...} if you do
%% not need to align the top of the contents to any
%% \cvsection title in the "main" bar.
\cvsection[page1sidebar]{Experience}

\cvevent{Software Engineer}{{\faBriefcase} SONY}{Aug 2014 -- Aug 2016}{Bengaluru, IN}
\begin{itemize}

\item Performing security scans, code enhancements \& bug fixes.
\item Production support \& troubleshooting tasks.
\item Working with Agile Scrum Methodologies.
\item Delivered sessions to internal teams on the state of the art technologies.
\end{itemize}

\divider

\cvsection[page1sidebar]{Projects}

\cvevent{}{{\faCode} Implementation \& Evaluation of Linux Queue Disciplines in ns-3}{Dec 2016 -- Ongoing}{Dublin, IN}
\begin{itemize}
\item Implementing a ns-3 model for a recently proposed AQM, PI2: A Linearized AQM for Classic and Scalable TCP.
\item AQM Mechanisms a.k.a. Queue Disciplines are designed to address the problem of Bufferbloat.
\end{itemize}

\cvevent{}{{\faCode} Improved Congestion Feedback in Data Center TCP}{Aug 2013 -- May 2014}{Mangalore, IN}
\begin{itemize}

\item DCTCP algorithm was modified to use mark from front strategy with an aim to reduce the Flow Completion Time of mice traffic.
\item Simulations results obtained using ns-2 confirmed the effectiveness of proposed algorithm.
\end{itemize}

\cvevent{}{{\faCode} System Hardening of Linux Box }{Oct 2013 -- Nov 2013}{Mangalore, IN}
\begin{itemize}

\item Implement multiple layer of defense mechanisms to lock down potential threats.
\item Worked on Iptables, TCP Wrappers, Port Knocking techniques and performing vulnerability scans.
\end{itemize}

\cvevent{}{{\faCode} Configuring Software Firewalls in Linux }{Feb 2013 -- Mar 2013}{Mangalore, IN}
\begin{itemize}
\item Implemented different rules and policies in iptables \& uncomplicated firewall (ufw).
\item Developed configurations to prevent attacks and block specific ports,
protocols, IP addresses.
\end{itemize}

\divider

\cvsection{Technical Skillset}

\cvtag{Python} 
\cvtag{Shell}
\cvtag{C++}
\cvtag{MySQL}

\divider\smallskip

\cvtag{Linux/Debian}
\cvtag{Kali Linux}
\cvtag{Windows}
\cvtag{Mac}

\divider\smallskip

\cvtag{Wireshark}
\cvtag{Nmap}
\cvtag{LaTex}
\cvtag{Git}
\cvtag{JIRA}
\cvtag{Chef}
\cvtag{ns-3}
\cvtag{Amazon Web Services}
\cvtag{DevOps}
\cvtag{Agile}
\cvtag{Performance Tuning}
\cvtag{Ethical Hacking}
\cvtag{Penetration Testing}
\cvtag{Network Architecture}



\clearpage

\cvsection[page2sidebar]{Publications}

\nocite{*}

\printbibliography[heading=pubtype,title={\printinfo{\faBook}{Books}},type=book]

Contributed a chapter \emph{"TCP Congestion Control in Data Center Networks".}
\divider

\printbibliography[heading=pubtype,title={\printinfo{\faGroup}{Conference Proceedings}},type=inproceedings]

%% If the NEXT page doesn't start with a \cvsection but you'd
%% still like to add a sidebar, then use this command on THIS
%% page to add it. The optional argument lets you pull up the 
%% sidebar a bit so that it looks aligned with the top of the
%% main column.
% \addnextpagesidebar[-1ex]{page3sidebar}


\end{document}
